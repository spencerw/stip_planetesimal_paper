\documentclass[onecolumn]{aastex63}
\usepackage{natbib}
%\definecolor{orcidlogocol}{HTML}{A6CE39}
\bibliographystyle{aasjournal}

\begin{document}

\title{PLANETESIMAL ACCRETION AT SHORT ORBITAL PERIODS}

\author{Spencer C. Wallace}
\affiliation{Astronomy Department, University of Washington, Seattle, WA 98195}

\author{Thomas R. Quinn}
\affiliation{Astronomy Department, University of Washington, Seattle, WA 98195}

\begin{abstract}
Formation models in which terrestrial bodies grow via the pairwise accretion of planetesimals have been reasonably successful at reproducing the general properties of the solar system, including small body populations. However, planetesimal accretion has not yet been fully explored in the context of more exotic terrestrial systems, particularly those that host short-period planets. In this work, we use direct N-body simulations to explore and understand the growth of planetary embryos from planetesimals in disks extending down to $\simeq$ 1 day orbital periods. We show that planetesimal accretion becomes nearly 100 percent efficient at short orbital periods, leading to embryo masses that are roughly twice as large as the classical isolation mass. For rocky bodies, the physical size of the object begins to occupy a significant fraction of its Hill sphere at orbital periods less than about 50 days. In this regime, most close encounters result in collisions, rather than scattering, and the system cannot bifurcate into a collection of dynamically hot planetesimals and dynamically cold oligarchs, like is seen in most models. The highly efficient accretion seen at short orbital periods implies that systems of tightly-packed inner planets should be almost completely devoid of any residual small bodies. We demonstrate the robustness of our results to assumptions about the initial disk model, and also investigate how far material can radially mix across the accretion boundary.
\end{abstract}

%\begin{list}
%\item Explain $\alpha$ controls importance of scattering vs accretion
%\item Also mention $\beta$. Introduces another relevant size scale, but we seem to get runaway growth no matter how large or small this is (is this true?)
%\item Show full disk vhi f6 and f4 simulations. Boundary of efficient accretion moves with density of bodies. Roughly lies where $\alpha = 0.1$.
%\item Eccentricities reach v = vesc regardless of alpha. Show narrow annulus simulations in which embryo 'cools' in planetesimal disk with varying $\alpha$.
%\item Show collision tree plot for full disk vhi f6. Do planetesimals move around much? How does motion of condensation fronts (due to pre-MS evolution of M stars) affect this?
%\item What about fragmentation?
%\item Planetesimals get completely consumed in inner disk. May have implications for planetesimal driven migration (outward, counteract type I migration)?
%\item Embryos form first in the inner disk. What are the implications of this? Might make outward planetesimal driven migration easier
%\end{list}

\section{Introduction} \label{sec:intro}

Planetesimal accretion is one of a number of stages in which micron-sized solids from the protostellar nebula coalesce to eventually build terrestrial planets. In the earliest stages, aerodynamic forces dominate the growth and evolution of the solids. Millimeter-sized bodies grow through adhesive pairwise collisions and stay well-coupled to the surrounding gas. Beyond this size, however, a number of growth barriers present themselves. Most notably, larger solids orbit the central star at Keplerian speeds as they decouple from the gas, which orbits at a sub-keplerian speed due to radial pressure support. This leads the solids to feel a headwind, which is maximally effective at sapping away angular momentum for objects around 1 meter in size. At this size, the timescale for the growing solids to fall onto the star is catastrophically short and leads to what is known as the drift barrier. In addition, two-body collisions between mm- to cm-sized bodies tend to result in bouncing or destruction, rather than continued growth. For these reasons, a number of mechanisms have been proposed which facilitate fast growth from mm to km sizes by locally concentrating solids. Dust traps, streaming instability, pebble piles...etc.

Beyond kilometer scales, gravity begins to dominate and aerodynamic gas drag plays a smaller and smaller role. During this phase, collision cross sections are enhanced as gravitational focusing (safronov citation) acts to bend the trajectories of bodies undergoing close encounters. Large bodies are most effective at focusing the trajectories of nearby planetesimals, leading to a period of runaway growth (citations to wetherill, kokubo+ida, barnes). Eventually, the largest bodies (known as oligarchs) dynamically heat the surrounding planetesimals, severely limiting further growth (cite kokubo+ida). The end result of this phase is a bimodal population of dynamically cold oligarchs, surrounded by dynamically hot, difficult to accrete residual planetesimals. Lines of evidence suggest that the asteroid belt, kuiper belt and the oort cloud are largely composed of the leftovers of this stage of planet formation. (Mention more specific evidence, morbidelli 09 paper, CAIs?)

Although gas drag has a minimal influence on the Moon to Mars-sized oligarchs, it is enough to prevent these largest bodies from perturbing each other onto crossing orbits. Simulations show that evaporation of the gas disk is required to allow instability to trigger a phase of giant impacts (Mention that disk fraction decay timescale roughly matches timing of giant impacts in SS). It is during this phase that oligarchs collide to form Earth-sized planets (chambers wetherill 1998, raymond 2006).

N body simulations allow us to work back to the planetesimal accretion phase. No one has applied this model to the in situ formation of STIPs yet. Does runaway and oligarchic growth still operate?

Outline of paper

Intro text goes here \citep{Wallace2019}

\section{Overview of Planetesimal Accretion}

\section{Methods}

\subsection{The Code}

\subsection{Initial Conditions}

\section{Results}

\subsection{Narrow Annulus}

\subsection{Full Disk}

\section{Simplifying Assumptions}

\subsection{Collision Cross Section}

\subsection{Collision Model}

\section{Summary and Discussion} \label{sec:discuss}

Summary and discussion text goes here

\bibliography{references}

\clearpage

\end{document}